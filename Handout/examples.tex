We give an implementation of the \pdm \ (Algorithm \ref{alg:pdip}) in \matlab, along with two example problems. One with equality constraints, one without.

\subsection{Numerical Examples}
\subsubsection{Example 1}
The effectivity of the algorithm can be demonstrated with the following explicit example of \eqref{eq:OptProblem}. For the problem defined by
\begin{gather*}
	Q = 2\cdot \begin{pmatrix}
	1 & 0\\ 0 & 2
\end{pmatrix}; \quad c = \vectortwo{1}{2}\\
\Aeq= \begin{pmatrix}
1 & -1
\end{pmatrix};\quad \beq= 0;\\
\Aineq = \begin{pmatrix}
0 & 1
\end{pmatrix}; \quad \beq= 10,
\end{gather*}
we choose backtracking parameters $ \alpha 0.05,\beta = 0.5 $ as well as $ \gamma = 0.1 $ for the reduction of the approximation parameter. Initial points are set to $ \xnull = (1,1)\trp, \lambda_0 = 1, \nu_0 = 0 $ and tolerances to $ \epsfeas = \epsopt = 10^{-4} $.\\

With this setup, the solver terminates after 20 iterations and yields the following results:
\begin{align*}
	\xopt &= \vectortwo{-1.0000}{-1.0000};\\
	\lambdaopt&= 5.6650\cdot 10^{-6};\\ \nu&= 2.7562\cdot 10^{-6};\\
	\fnull(\xopt) &= -1.5000;\\ \etahat&= 6.2315\cdot 10^{-6}.
\end{align*}




\subsubsection{Example 2 }
To test algorithm \eqref{alg:pdip} on a problem without equality constraints, we moreover provide a second example, taken from \cite{EX}. The problem is defined
\begin{gather*}
		Q = 2\cdot \begin{pmatrix}
	3 & 1\\ 1 & 1
	\end{pmatrix}; \quad c = \vectortwo{1}{6};\\
	\Aineq = \begin{pmatrix}
	-2 & -3 \\ 0& -1
	\end{pmatrix}; \quad \bineq= \begin{pmatrix}
	-4\\0\\0
	\end{pmatrix},
\end{gather*}
while initial values are chosen as $ \xnull = (1,1)\trp $ and $ \lambda_0 = (0.2,1,1)\trp $ (like suggested in remark \ref{re:dualinit}). Tolerances and backtracking-paramters are chosen as in Example 1.

With termination after 20 iterations, the solver yields the results
\begin{align}
	\xopt &= \vectortwo{0.5000}{1.0000}\\
	\lambdaopt &= \vectorthree{3.0000}{0.0001}{2.812\cdot 10^{-5}}\\
	\fnull(\xopt) &= 9.2500.
\end{align}
