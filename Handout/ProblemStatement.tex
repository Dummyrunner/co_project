For theoretical discussions, we consider the convex constrained optimization problem
\begin{equation}
\OptProblem
\label{eq:OptProblem}
\end{equation}
with $ f_0:\Rn \longrightarrow \R $ convex and twice differentiable, $ f_i:\Rn \longrightarrow \R $ for $ i=1,\dots,m $ convex and differentiable,  with equality and inequality constraints described by
$ \Aeq \in \R^{n\times p}, \beq \in  \R^p $. For such an optimization problem, we call its Lagrangian $ L:\Rn \times \Rm \times \Rp \longrightarrow \R $
with  \[ L(x,\lambda,\nu) = \fnull(x) + \lambda\trp f(x) + \nu \trp (\Aeq x - \beq).  \]
Further, we denote its dual problem by
\begin{equation}
	\OptProblemDual
	\label{eq:OptProblemDual}
\end{equation}
with \[ g(\lambda, \nu) = \inf_{x\in \Rn} L(x,\lambda,\nu). \]


Moreover, we give a \matlab-implementation of a primal-dual interiorpoint method for  convex quadratic optimization problems. Quadratic problems are a subclass of \eqref{eq:OptProblem} and denote as
\begin{equation}
	\OptProblemquad
	\label{eq:QuadProblem}
\end{equation}
with matrices $ 0 \prec Q\in \R^{n \times n}, c\in \Rn $.
Quadratic programming has multiple applications in various fields, such as artificial intelligence or control. To pick one example, a way to apply it on  a zero terminal constraint \mpclong \ problem is shown in the appendix.