\subsection{Concept of Barrier Methods}
Convex optimization Problems with no inequality constraints can  be solved efficiently by using Newton's method. If inequality constraints are involved, Newton's method can not guarantee feasibiliy of a solution. It is hence desirable, to transform an inequality-constrained optimization problem into a only equality-constrained one. Therefore, we move the inequality constraints implicitley to the objective function. A simple and also precise way to do this, evaluate an  indicator function  
\begin{align}
	\ifct (x) :=
	\begin{cases}
		0 \quad &\text{for } u \neq 0\\
		\infty &\text{for } u > 0
	\end{cases}
\end{align}
on the values of the inequality constraints $ f_i, i=1,\dots,m $. Then, the optimization Problem has the shape
\begin{align}
	\begin{aligned}
	& \underset{x}{\text{minimize}}
	& & f_0(x) + \sum_{i=1}^{m} \ifct(f_i(x))\\
	& \text{subject to}
	& & g_i(x) = 0, \; i = 1,\dots, p.
	\end{aligned}
\end{align}
This problem is an equivalent to\ref{eq:OptProblem} and has no inequality constraints. However, it is clearly neither convex nor continuous (and hence not differentiable). Since we need these properties to solve the optimization problem computationally, we approximate the indicator function $ \ifct $ with the function
\begin{equation}
	\ifcthat (u) =
	\begin{cases}
	 \frac{1}{t}\log(-u) \quad \text{for } u < 0,\\
	 \infty \quad \text{for } u\geq 0,
	\end{cases}
\end{equation}
The parameter $ t>0 $ sets the approximation's accuracy. The higher $ t $ is, the better the indicator function is approximated.
%TODO evt plot von Ihat einfügen!


%TODO ápprox problem

Note, that $ \frac{1}{t}\log(-u) $ is convex, increasing in $ u $, and differentiable on the admissible set. Hence the entire 


\subsection{Newton's Method}
Newton's method is an iterative process to solve nonlinear equality systems
\begin{equation}
\Fx= 0
\end{equation}
for a differentiable map $ F: \Rn \longrightarrow \Rm $. The idea of this algorithm is as follows: At a given point $ \xk $, the zero of the linear approximation of $ F $ around $ \xk $  is computed. This point is chosen as the next iterate $ \xkplus $. In particular, a linear approximation of $ F $ in  $ \xk $ is defined as
\begin{equation}
	\Lx:= \Fxk + \jacFxk(x - \xk) \text{ for } x\in \Rn,
\end{equation}
where $ \jacFxk $ is the Jacobian of $ F $ at the point $ \xk $. If $ \jacFxk $ invertible, the point $ \xtil $ with $ L(\xtil)=0 $ is exactly the solution of  the linear equality $ \jacFxk x = -\Fx $.
%Though Newton's method is in general not guaranteed to converge,  
Technical conditions and proofs about convergence rates of Newton's method can be found in \cite{SO}.

%TODO Newton algo angeben??
%\begin{algorithm}
%	\SetAlgoLined
%	\KwResult{$ \xtil $, approximate solution of nonlinear system of equalities $ \Fx = 0$ }
%	initialization\; Function $ F: \Rn \longrightarrow \Rn $, initial point $ \xnull $\\
%	\While{While condition}{
%		instructions\;
%		\eIf{condition}{
%			instructions1\;
%			instructions2\;
%		}{
%			instructions3\;
%		}
%	}
%	\caption{Newton's Method}
%\end{algorithm}

For the purpose of optimizing a convex, twice differentiable objective function $ \fnull $ we want to find a zero of the gradient $ \grad \fnull $. Therefore we can apply the Newton Method to solve the non-linear equation $ \Fx := \grad \fnull (x) = 0 $. By convexity, satisfying $ \grad \fnull (\xopt) = 0$ is not only neccessary, but also sufficient for $ \xopt $ to be a global minimum of $ \fnull $.

Present main theorems/algorithm. Explain idea, explain algorithm, 
provide a convergence proof, discuss main properties (advantages and disadvantages)
%\begin{algorithm}
%	
%\end{algorithm}
Use algorithm environment in Latex to present algorithm (pseudo-code)