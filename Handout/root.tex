%%%%%%%%%%%%%%%%%%%%%%%%%%%%%%%%%%%%%%%%%%%%%%%%%%%%%%%%%%%%%%%%%%%%%%%%%%%%%%%%
%2345678901234567890123456789012345678901234567890123456789012345678901234567890
%        1         2         3         4         5         6         7         8

\documentclass[letterpaper, 10 pt, conference]{ieeeconf}  % Comment this line out if you need a4paper

%\documentclass[a4paper, 10pt, conference]{ieeeconf}      % Use this line for a4 paper

\IEEEoverridecommandlockouts                              % This command is only needed if 
                                                          % you want to use the \thanks command

\overrideIEEEmargins                                      % Needed to meet printer requirements.

% See the \addtolength command later in the file to balance the column lengths
% on the last page of the document


% MYTHINGS
\usepackage{amssymb,amsmath, amsfonts,color}
\usepackage[ruled, vlined]{algorithm2e}
\newtheorem{theorem}{Theorem}
\newtheorem{definition}{Definition}
\newtheorem{remark}{Remark}
\newtheorem{lemma}{Lemma}
\newtheorem{corollary}{Corollary}
\newcommand{\grad}{\nabla}
\newcommand{\R}{\mathbb{R}}
\newcommand{\Rn}{\mathbb{R}^n}
\newcommand{\Rm}{\mathbb{R}^m}
\newcommand{\Fx}{F(x)}
\newcommand{\jacF}{J F}
\newcommand{\xk}{x_k}
\newcommand{\Fxk}{F(\xk)}
\newcommand{\jacFxk}{\jacF (\xk)}
\newcommand{\xkplus}{x_{k+1}}
\newcommand{ \Lx}{L(x)}
\newcommand{\xtil}{\tilde{x}}
\newcommand{\fnull}{f_0}
\newcommand{\todo}{TODO!}
\newcommand{\xnull}{x_0}
\newcommand{\ind}[2]{{#1}_{\mathrm{#2}}}
\newcommand{\ifct}{\ind{I}{-}}
\newcommand{\xopt}{x^*}
\newcommand{\OptProblem}{
	\begin{aligned}
	& \underset{x}{\text{minimize}}
	& & f_0(x) \\
	& \text{subject to}
	& & f_i(x) \leq 0, \; i = 1, \ldots, m.\\
	& & &g_i(x) = 0, \; i = 1,\dots, p.
	\end{aligned}
	}
\newcommand{\OptProblemquad}{
	\begin{aligned}
		& \underset{x}{\text{minimize}}
		& & f_0(x) \todo auf quad anpassen!!\\
		& \text{subject to}
		& & f_i(x) \leq 0, \; i = 1, \ldots, m.\\
		& & &g_i(x) = 0, \; i = 1,\dots, p.
	\end{aligned}
}
% The following packages can be found on http:\\www.ctan.org
%\usepackage{graphics} % for pdf, bitmapped graphics files
%\usepackage{epsfig} % for postscript graphics files
%\usepackage{mathptmx} % assumes new font selection scheme installed
%\usepackage{times} % assumes new font selection scheme installed
\usepackage{amsmath} % assumes amsmath package installed
\usepackage{amssymb}  % assumes amsmath package installed
%
\title{\LARGE \bf Magic Gradient Descent*
}


\author{Albert Author$^{1}$ and Bernard D. Researcher$^{2}$% <-this % stops a space
\thanks{*Project within the course Convex Optimization, University of Stuttgart, \today.}% <-this % stops a space
\thanks{$^{1}$Albert Author is a student of the Bachelor study program Mechatronics, University of Stuttgart,
        {\tt\small albert.author@papercept.net}}%
\thanks{$^{2}$Bernard D. Researcher is a student of the Master study program Engineering ... ., University of Stuttgart,
        {\tt\small b.d.researcher@ieee.org}}%
}


\begin{document}



\maketitle
\thispagestyle{empty}
\pagestyle{empty}


%%%%%%%%%%%%%%%%%%%%%%%%%%%%%%%%%%%%%%%%%%%%%%%%%%%%%%%%%%%%%%%%%%%%%%%%%%%%%%%%
\begin{abstract}

Describe in a few sentences what the paper is about and why it is interesting 
to read it.

\end{abstract}


%%%%%%%%%%%%%%%%%%%%%%%%%%%%%%%%%%%%%%%%%%%%%%%%%%%%%%%%%%%%%%%%%%%%%%%%%%%%%%%%
\section{INTRODUCTION}

Some general introducing sentences about the topic, motivation and relevance of problem/algorithm.

In this paper we give an introduction to the results presented in paper(s) \cite{Bro-14}.



We present the problem statement (optimization problem)
the main results/algorithms, discuss the underlying ideas and illustrate the results 
by numerical simulations.

Notation. Define notation.

%%%%%%%%%%%%%%%%%%%%%%%%%%%%%%%%%%%%%%%%%%%%%%%%%%%%%%%%%%%%%%%%%%%%%%%%%%%%%%%%
\section{PROBLEM STATEMENT AND BACKGROUND}

Provide a mathematical problem description. If necessary, some background material.

\begin{equation}
\OptProblem
\label{eq:OptProblem}
\end{equation}

-convex quadratic opt prblem with inequality constraints\\
-> how to handle ineq. constraints?
-technical relevance: optimal control, mpc


%%%%%%%%%%%%%%%%%%%%%%%%%%%%%%%%%%%%%%%%%%%%%%%%%%%%%%%%%%%%%%%%%%%%%%%%%%%%%%%%
\section{MAIN RESULTS}
\subsection{Concept of Barrier Methods}
Convex optimization Problems with no inequality constraints can  be solved efficiently by using Newton's method. If inequality constraints are involved, Newton's method can not guarantee feasibiliy of a solution. It is hence desirable, to transform an inequality-constrained optimization problem into a only equality-constrained one. Therefore, we move the inequality constraints implicitley to the objective function. A simple and also precise way to do this, evaluate an  indicator function  
\begin{align}
	\ifct (x) :=
	\begin{cases}
		0 \quad &\text{for } u \neq 0\\
		\infty &\text{for } u > 0 b
	\end{cases}
\end{align}
on the values of the inequality constraints $ f_i, i=1,\dots,m $. Then, the optimization Problem has the shape
\begin{align}
	\begin{aligned}
	& \underset{x}{\text{minimize}}
	& & f_0(x) + \sum_{i=1}^{m} \ifct(f_i(x))\\
	& \text{subject to}
	& & g_i(x) = 0, \; i = 1,\dots, p.
	\end{aligned}
\end{align}
This problem is an equivalent to\ref{eq:OptProblem} and has no inequality constraints. However, it is clearly neither convex nor continuous (and hence not differentiable).

\subsection{Newton's Method}
Newton's method is an iterative process to solve nonlinear equality systems
\begin{equation}
\Fx= 0
\end{equation}
for a differentiable map $ F: \Rn \longrightarrow \Rm $. The idea of this algorithm is as follows: At a given point $ \xk $, the zero of the linear approximation of $ F $ around $ \xk $  is computed. This point is chosen as the next iterate $ \xkplus $. In particular, a linear approximation of $ F $ in  $ \xk $ is defined as
\begin{equation}
	\Lx:= \Fxk + \jacFxk(x - \xk) \text{ for } x\in \Rn,
\end{equation}
where $ \jacFxk $ is the Jacobian of $ F $ at the point $ \xk $. If $ \jacFxk $ invertible, the point $ \xtil $ with $ L(\xtil)=0 $ is exactly the solution of  the linear equality $ \jacFxk x = -\Fx $.
%Though Newton's method is in general not guaranteed to converge,  
Technical conditions and proofs about convergence rates of Newton's method can be found in \cite{SO}.

%TODO Newton algo angeben??
%\begin{algorithm}
%	\SetAlgoLined
%	\KwResult{$ \xtil $, approximate solution of nonlinear system of equalities $ \Fx = 0$ }
%	initialization\; Function $ F: \Rn \longrightarrow \Rn $, initial point $ \xnull $\\
%	\While{While condition}{
%		instructions\;
%		\eIf{condition}{
%			instructions1\;
%			instructions2\;
%		}{
%			instructions3\;
%		}
%	}
%	\caption{Newton's Method}
%\end{algorithm}

For the purpose of optimizing a convex, twice differentiable objective function $ \fnull $ we want to find a zero of the gradient $ \grad \fnull $. Therefore we can apply the Newton Method to solve the non-linear equation $ \Fx := \grad \fnull (x) = 0 $. By convexity, satisfying $ \grad \fnull (\xopt) = 0$ is not only neccessary, but also sufficient for $ \xopt $ to be a global minimum of $ \fnull $.

Present main theorems/algorithm. Explain idea, explain algorithm, 
provide a convergence proof, discuss main properties (advantages and disadvantages)
%\begin{algorithm}
%	
%\end{algorithm}
Use algorithm environment in Latex to present algorithm (pseudo-code)

%%%%%%%%%%%%%%%%%%%%%%%%%%%%%%%%%%%%%%%%%%%%%%%%%%%%%%%%%%%%%%%%%%%%%%%%%%%%%%%%
\section{EXAMPLES}

Show and discuss simulation examples etc....



%%%%%%%%%%%%%%%%%%%%%%%%%%%%%%%%%%%%%%%%%%%%%%%%%%%%%%%%%%%%%%%%%%%%%%%%%%%%%%%%
\section{CONCLUSIONS}

Summarize the main points (with more details than in the preceding introduction).
The paper should not be between 4 and 8 pages.

%%%%%%%%%%%%%%%%%%%%%%%%%%%%%%%%%%%%%%%%%%%%%%%%%%%%%%%%%%%%%%%%%%%%%%%%%%%%%%%%


\addtolength{\textheight}{-12cm}   % This command serves to balance the column lengths
                                  % on the last page of the document manually. It shortens
                                  % the textheight of the last page by a suitable amount.
                                  % This command does not take effect until the next page
                                  % so it should come on the page before the last. Make
                                  % sure that you do not shorten the textheight too much.

%%%%%%%%%%%%%%%%%%%%%%%%%%%%%%%%%%%%%%%%%%%%%%%%%%%%%%%%%%%%%%%%%%%%%%%%%%%%%%%%


%%%%%%%%%%%%%%%%%%%%%%%%%%%%%%%%%%%%%%%%%%%%%%%%%%%%%%%%%%%%%%%%%%%%%%%%%%%%%%%%
\section*{APPENDIX}

Add for example your Matlab code here. (Code should be nicely formated and documented).

Appendixes should appear before the acknowledgment.

\section*{ACKNOWLEDGMENT}


%%%%%%%%%%%%%%%%%%%%%%%%%%%%%%%%%%%%%%%%%%%%%%%%%%%%%%%%%%%%%%%%%%%%%%%%%%%%%%%%

%\bibliographystyle{plain}
%\bibliography{c:/home/Lehre/VO/SS17/ND/Projects/Latex/mybib}

\begin{thebibliography}{99}
	  \bibitem{SO} Carsten Scherer {\em Vorlesungsskript Einführung in die Optimierung} 2019: Lehrstuhl für Mathematische Systemtheorie, Universität Stuttgart.
	  \bibitem{BV} Stephen Boyd, Lieven Vandenberghe {\em Convex Optimization} 2004: Cambridge University Press.
\end{thebibliography}

\end{document}
