\subsection{\matlab-Implementation}
In the following, the \matlab \ code of the primal-dual interior point method algorithm is presented\\

Algorithm main function:\\
{\tiny \lstinputlisting[style=Matlab-editor]{../code/ipquad_pd.m}}
Newton step including line-search:\\
{\tiny \lstinputlisting[style=Matlab-editor]{../code/newtonquad_pd.m}}
Computing KKT residual vector from problem matrices and current points:\\
{\tiny \lstinputlisting[style=Matlab-editor]{../code/res_kkt.m}}



\subsection{\mpclong}
To illustrate the practical relevance of quadratic programming in e.g. control tasks, we here present the application of interior methods on a \mpclong \ (\mpc) problem. More precisely, a \mpc \ applied on a linear system with zero terminal constraint (ZTC).
We consider the discrete time linear system
\begin{align}
\xkplus = \Ad \xk + \Bd \uk, \label{eq:dynsys}
\end{align}
where $ \xk \in \Rn, \uk \in \Rm $ for all $ k \in  \N_0$, with the constraints that $ \normmax{\uk} \leq \boundu$
and $ \normmax{\xk} \leq \boundx $ for all steps $ k\in \N_0 $. Further we assume that \eqref{eq:dynsys} has an equilibrium in $ x = 0 $ and the current state $ \xnull $ is given. The maximum-norm is defined as the maximum absolute value over all entries of the vector, $\normmax{x} = \max_{1\leq i\leq n } |x_i| $ for $ x\in \Rn $.
Goal is to find a input signal that steers the internal state $ x $ to zero, while additionally keeping $ u $ as small as possible. Therefore, we consider the next $ N \in \N $ timesteps. We call $ N $ the prediction horizon.
This leads to optimization of a certain objective function over all possible predicted steering signals $ \ubar = ({\ubar_1},{\dots},{\ubar_{N-1}})\trp $. We simulate the system for the next $ N $ timestep, hence we consider the sequence of states arising from applying a predicted sequence of input signals $ \ubar $. The sequence of predicted states we denote as $ (\xnullbar,\dots,\xbar_N)\trp $.
We choose the quadratic objective function
\begin{equation}
\sum_{k = 0}^{N-1} \underbrace{\xbark\trp Q \xbark }_{=: \norm{\xbark}_Q}+ \underbrace{\ubark \trp R \ubark}_{=:\norm{\ubark}_R} \label{eq:mpcobjraw}
\end{equation}
under the condition, that $ \xbar_N $, the last state in the predicted time, equals zero,
to minimize. The regarding predicted states directly follow from the system dynamics, in particular
\[ \xbar_{k+1} = \Ad \xbark + \Bd \ubark. \] After finding an optimal signal $ \ubar $ we apply $ \ubar_0 $ in the next time step. After this first step, we again start an optimization over the next $ N $ timesteps. So even though the optimization problem is computed considering all signals up to the prediction horizon, the result is only applied for one timestep, before the next optimization is executed.

We can summarize one optimization step as an optimization problem
\begin{equation}
\begin{aligned} \label{eq:mpcProblem}
& \underset{\ubar=(\ubar_0,\dots,\ubar_{N-1})\trp}{\text{minimize}}
& & \sum_{k = 1}^{N-1} \norm{\xbark}_Q + \norm{\xbark}_R\\
& \text{subject to}
& & \xbar_0 = x_0,\\
& & &\xbar_N = 0,\\
& & &\xbar_{k+1} = \Ad \xbark + \Bd \ubark \text{\quad for all } k = 0,\dots,N,\\
& & &\norm{\xbark} \leq \boundx, \norm{\ubark} \leq \boundu \text{\quad for all } k = 0,\dots,N.
\end{aligned}
\end{equation}


This problem can be transcribed into the form of $ \eqref{eq:OptProblem} $. We therefore optimize over the whole vector \[ \xtilde := \vectortwo{\xbar}{\ubar}. \] The objective function \eqref{eq:mpcobjraw} then can  be written as
\[ \fnull(\xtilde) = \xtilde \trp  H \xtilde, \] with  the block diagonal matrix \[ H = \mathrm{diag}(\underbrace{Q,\dots,Q}_{N\cdot n \text{ blocks}},\underbrace{0}_{n \times n},\underbrace{R,\dots,R}_{N \cdot n \text{ blocks}}.) \]
We further formulate the constraints at the maximum norms of state and input as $ \Aineq \xtilde \leq \bineq $, with $ \bineq = \ones \in \R^{N(n+m)+n \times 1} $ and
\[ \Aineq = \begin{pmatrix}
I_{n(N+1)} & 0\\
-I_{n(N+1)} & 0\\
0 & I_{N\cdot m}\\
0 & -I_{N\cdot m}
\end{pmatrix}\]
where the indices of the identity-matrix $ I $ denote its dimensional size.
Rearranging the system dynamics \eqref{eq:dynsys} and taking inital condition as well as zero terminal constraint into account can be written as the equality constraints $ \Aeq \xtilde = \beq $ with
$ \Aeq =   \left( \Aeqx \quad \Aequ \right), $ where

\begin{gather*}
\Aeqx =  \begin{pmatrix}
I_{n\times n} & 0 & \cdots & \cdots & 0\\
\Ad & -I_{n\times n} & \cdots & \cdots & \vdots\\
0 & \ddots & \ddots &  &  0\\
\vdots & & \ddots  & \ddots & 0\\
\vdots & \cdots & & \Ad & -I_{n\times n}\\
0 &\cdots& \cdots & 0& I_{n\times n},
\end{pmatrix}\\
\Aequ = \begin{pmatrix}
0 & \cdots & 0\\
\Bd &\ddots &\vdots\\
\vdots &\ddots & 0\\
0 & \cdots & \Bd\\
0 & \cdots & 0
\end{pmatrix}.
\end{gather*}

With these matrices, we rewrote \eqref{eq:mpcProblem} as an equivalent problem in shape of  \eqref{eq:OptProblem}.